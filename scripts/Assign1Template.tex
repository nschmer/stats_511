\documentclass[]{article}
\usepackage{lmodern}
\usepackage{amssymb,amsmath}
\usepackage{ifxetex,ifluatex}
\usepackage{fixltx2e} % provides \textsubscript
\ifnum 0\ifxetex 1\fi\ifluatex 1\fi=0 % if pdftex
  \usepackage[T1]{fontenc}
  \usepackage[utf8]{inputenc}
\else % if luatex or xelatex
  \ifxetex
    \usepackage{mathspec}
  \else
    \usepackage{fontspec}
  \fi
  \defaultfontfeatures{Ligatures=TeX,Scale=MatchLowercase}
\fi
% use upquote if available, for straight quotes in verbatim environments
\IfFileExists{upquote.sty}{\usepackage{upquote}}{}
% use microtype if available
\IfFileExists{microtype.sty}{%
\usepackage{microtype}
\UseMicrotypeSet[protrusion]{basicmath} % disable protrusion for tt fonts
}{}
\usepackage[margin=1in]{geometry}
\usepackage{hyperref}
\hypersetup{unicode=true,
            pdftitle={Assign1Template},
            pdfborder={0 0 0},
            breaklinks=true}
\urlstyle{same}  % don't use monospace font for urls
\usepackage{color}
\usepackage{fancyvrb}
\newcommand{\VerbBar}{|}
\newcommand{\VERB}{\Verb[commandchars=\\\{\}]}
\DefineVerbatimEnvironment{Highlighting}{Verbatim}{commandchars=\\\{\}}
% Add ',fontsize=\small' for more characters per line
\usepackage{framed}
\definecolor{shadecolor}{RGB}{248,248,248}
\newenvironment{Shaded}{\begin{snugshade}}{\end{snugshade}}
\newcommand{\AlertTok}[1]{\textcolor[rgb]{0.94,0.16,0.16}{#1}}
\newcommand{\AnnotationTok}[1]{\textcolor[rgb]{0.56,0.35,0.01}{\textbf{\textit{#1}}}}
\newcommand{\AttributeTok}[1]{\textcolor[rgb]{0.77,0.63,0.00}{#1}}
\newcommand{\BaseNTok}[1]{\textcolor[rgb]{0.00,0.00,0.81}{#1}}
\newcommand{\BuiltInTok}[1]{#1}
\newcommand{\CharTok}[1]{\textcolor[rgb]{0.31,0.60,0.02}{#1}}
\newcommand{\CommentTok}[1]{\textcolor[rgb]{0.56,0.35,0.01}{\textit{#1}}}
\newcommand{\CommentVarTok}[1]{\textcolor[rgb]{0.56,0.35,0.01}{\textbf{\textit{#1}}}}
\newcommand{\ConstantTok}[1]{\textcolor[rgb]{0.00,0.00,0.00}{#1}}
\newcommand{\ControlFlowTok}[1]{\textcolor[rgb]{0.13,0.29,0.53}{\textbf{#1}}}
\newcommand{\DataTypeTok}[1]{\textcolor[rgb]{0.13,0.29,0.53}{#1}}
\newcommand{\DecValTok}[1]{\textcolor[rgb]{0.00,0.00,0.81}{#1}}
\newcommand{\DocumentationTok}[1]{\textcolor[rgb]{0.56,0.35,0.01}{\textbf{\textit{#1}}}}
\newcommand{\ErrorTok}[1]{\textcolor[rgb]{0.64,0.00,0.00}{\textbf{#1}}}
\newcommand{\ExtensionTok}[1]{#1}
\newcommand{\FloatTok}[1]{\textcolor[rgb]{0.00,0.00,0.81}{#1}}
\newcommand{\FunctionTok}[1]{\textcolor[rgb]{0.00,0.00,0.00}{#1}}
\newcommand{\ImportTok}[1]{#1}
\newcommand{\InformationTok}[1]{\textcolor[rgb]{0.56,0.35,0.01}{\textbf{\textit{#1}}}}
\newcommand{\KeywordTok}[1]{\textcolor[rgb]{0.13,0.29,0.53}{\textbf{#1}}}
\newcommand{\NormalTok}[1]{#1}
\newcommand{\OperatorTok}[1]{\textcolor[rgb]{0.81,0.36,0.00}{\textbf{#1}}}
\newcommand{\OtherTok}[1]{\textcolor[rgb]{0.56,0.35,0.01}{#1}}
\newcommand{\PreprocessorTok}[1]{\textcolor[rgb]{0.56,0.35,0.01}{\textit{#1}}}
\newcommand{\RegionMarkerTok}[1]{#1}
\newcommand{\SpecialCharTok}[1]{\textcolor[rgb]{0.00,0.00,0.00}{#1}}
\newcommand{\SpecialStringTok}[1]{\textcolor[rgb]{0.31,0.60,0.02}{#1}}
\newcommand{\StringTok}[1]{\textcolor[rgb]{0.31,0.60,0.02}{#1}}
\newcommand{\VariableTok}[1]{\textcolor[rgb]{0.00,0.00,0.00}{#1}}
\newcommand{\VerbatimStringTok}[1]{\textcolor[rgb]{0.31,0.60,0.02}{#1}}
\newcommand{\WarningTok}[1]{\textcolor[rgb]{0.56,0.35,0.01}{\textbf{\textit{#1}}}}
\usepackage{graphicx,grffile}
\makeatletter
\def\maxwidth{\ifdim\Gin@nat@width>\linewidth\linewidth\else\Gin@nat@width\fi}
\def\maxheight{\ifdim\Gin@nat@height>\textheight\textheight\else\Gin@nat@height\fi}
\makeatother
% Scale images if necessary, so that they will not overflow the page
% margins by default, and it is still possible to overwrite the defaults
% using explicit options in \includegraphics[width, height, ...]{}
\setkeys{Gin}{width=\maxwidth,height=\maxheight,keepaspectratio}
\IfFileExists{parskip.sty}{%
\usepackage{parskip}
}{% else
\setlength{\parindent}{0pt}
\setlength{\parskip}{6pt plus 2pt minus 1pt}
}
\setlength{\emergencystretch}{3em}  % prevent overfull lines
\providecommand{\tightlist}{%
  \setlength{\itemsep}{0pt}\setlength{\parskip}{0pt}}
\setcounter{secnumdepth}{0}
% Redefines (sub)paragraphs to behave more like sections
\ifx\paragraph\undefined\else
\let\oldparagraph\paragraph
\renewcommand{\paragraph}[1]{\oldparagraph{#1}\mbox{}}
\fi
\ifx\subparagraph\undefined\else
\let\oldsubparagraph\subparagraph
\renewcommand{\subparagraph}[1]{\oldsubparagraph{#1}\mbox{}}
\fi

%%% Use protect on footnotes to avoid problems with footnotes in titles
\let\rmarkdownfootnote\footnote%
\def\footnote{\protect\rmarkdownfootnote}

%%% Change title format to be more compact
\usepackage{titling}

% Create subtitle command for use in maketitle
\providecommand{\subtitle}[1]{
  \posttitle{
    \begin{center}\large#1\end{center}
    }
}

\setlength{\droptitle}{-2em}

  \title{Assign1Template}
    \pretitle{\vspace{\droptitle}\centering\huge}
  \posttitle{\par}
    \author{}
    \preauthor{}\postauthor{}
    \date{}
    \predate{}\postdate{}
  

\begin{document}
\maketitle

\#1. Use the data described in Problem 3.30 regarding lumber. From the
files you downloaded above, you will find the data under CH03, named
ex3-30.txt. Use the following commands to import and summarize the data.

\begin{Shaded}
\begin{Highlighting}[]
\CommentTok{#Use path name with R Markdown}
\CommentTok{#Laptop}
\CommentTok{# lumber <- read.csv("~/Documents/GitHub/CSU/stats_511/data/assignment_1/ASCII-comma/CH03/ex3-30.TXT", quote = "'")}

\CommentTok{#office}
\NormalTok{lumber <-}\StringTok{ }\KeywordTok{read.csv}\NormalTok{(}\StringTok{"/Users/natalieschmer/Desktop/GitHub/stats_511/data/ASCII-comma/CH03/ex3-30.TXT"}\NormalTok{, }\DataTypeTok{quote =} \StringTok{"'"}\NormalTok{)}

\CommentTok{#View(lumber)  this doesn't work well with R Markdown}
\KeywordTok{str}\NormalTok{(lumber)}
\end{Highlighting}
\end{Shaded}

\begin{verbatim}
## 'data.frame':    70 obs. of  1 variable:
##  $ Number: int  7 8 6 4 9 11 9 9 9 10 ...
\end{verbatim}

\textbf{A. Include the histogram in your assignment.}

\begin{Shaded}
\begin{Highlighting}[]
\KeywordTok{hist}\NormalTok{(lumber}\OperatorTok{$}\NormalTok{Number, }
     \DataTypeTok{main =} \StringTok{"Histogram of Lumber"}\NormalTok{,}
     \DataTypeTok{xlab =} \StringTok{"Lumber"}\NormalTok{,}
     \DataTypeTok{ylab =} \StringTok{"Frequency"}\NormalTok{)}
\end{Highlighting}
\end{Shaded}

\includegraphics{Assign1Template_files/figure-latex/unnamed-chunk-1-1.pdf}
\textbf{B. Give the mean and median of the sample.}

\begin{Shaded}
\begin{Highlighting}[]
\NormalTok{(}\KeywordTok{mean}\NormalTok{(lumber}\OperatorTok{$}\NormalTok{Number))}
\end{Highlighting}
\end{Shaded}

\begin{verbatim}
## [1] 7.728571
\end{verbatim}

\begin{Shaded}
\begin{Highlighting}[]
\KeywordTok{median}\NormalTok{(lumber}\OperatorTok{$}\NormalTok{Number)}
\end{Highlighting}
\end{Shaded}

\begin{verbatim}
## [1] 8
\end{verbatim}

\textbf{C. Does the data appear to be ``normal'' (bell-shaped)? Justify
your response based on your histogram from above.}

Yes, I think the data could be considred normal due to the fact that
there is only 1 majority of the observations fall in the 7-9 range,
which matches the median and mean, with decreasing frequency moving away
from the middle.

\#2 Survival Times

\begin{Shaded}
\begin{Highlighting}[]
\CommentTok{#office}
\NormalTok{survival <-}\StringTok{ }\KeywordTok{read.csv}\NormalTok{(}\StringTok{"/Users/natalieschmer/Desktop/GitHub/stats_511/data/ASCII-comma/CH03/ex3-7.TXT"}\NormalTok{, }\DataTypeTok{quote =} \StringTok{"'"}\NormalTok{)}
\KeywordTok{str}\NormalTok{(survival)}
\end{Highlighting}
\end{Shaded}

\begin{verbatim}
## 'data.frame':    28 obs. of  2 variables:
##  $ StandardTherapy: int  4 14 29 6 15 2 6 13 24 16 ...
##  $ NewTherapy     : int  5 17 27 9 20 15 14 18 29 19 ...
\end{verbatim}

\textbf{A. What is the sample mean and sample standard deviation for
each of the therapies?}

\begin{Shaded}
\begin{Highlighting}[]
\CommentTok{#Standard Therapy}
\KeywordTok{mean}\NormalTok{(survival}\OperatorTok{$}\NormalTok{StandardTherapy)}
\end{Highlighting}
\end{Shaded}

\begin{verbatim}
## [1] 15.67857
\end{verbatim}

\begin{Shaded}
\begin{Highlighting}[]
\KeywordTok{sd}\NormalTok{(survival}\OperatorTok{$}\NormalTok{StandardTherapy)}
\end{Highlighting}
\end{Shaded}

\begin{verbatim}
## [1] 9.630405
\end{verbatim}

\begin{Shaded}
\begin{Highlighting}[]
\CommentTok{#New therapy}
\KeywordTok{mean}\NormalTok{(survival}\OperatorTok{$}\NormalTok{NewTherapy)}
\end{Highlighting}
\end{Shaded}

\begin{verbatim}
## [1] 20.71429
\end{verbatim}

\begin{Shaded}
\begin{Highlighting}[]
\KeywordTok{sd}\NormalTok{(survival}\OperatorTok{$}\NormalTok{NewTherapy)}
\end{Highlighting}
\end{Shaded}

\begin{verbatim}
## [1] 9.808753
\end{verbatim}

\textbf{B. Construct side-by-side boxplots showing the survival times
for each therapy.}

\begin{Shaded}
\begin{Highlighting}[]
\KeywordTok{boxplot}\NormalTok{(survival, }\DataTypeTok{xlab =} \StringTok{"Therapy type"}\NormalTok{, }\DataTypeTok{ylab=} \StringTok{"Survival Times"}\NormalTok{)}
\end{Highlighting}
\end{Shaded}

\includegraphics{Assign1Template_files/figure-latex/unnamed-chunk-5-1.pdf}


\end{document}
